\documentclass{beamer}

\usepackage[utf8]{inputenc}
\usepackage[ngerman]{babel}
\usepackage{amsmath,amsfonts,amssymb}
\usepackage{url}

\setbeamertemplate{navigation symbols}{}
\setbeamertemplate{footline}[frame number]

\usetheme{Dresden}
\usecolortheme{rose}

\title[Circuit Architecture: Embryonics Approach]
{Bio-inspired circuit architectures\\- The Embryonics Approach -}

\author[Frank Lange]{Frank Lange\\
\ \newline
\tiny \url{langefrq@informatik.hu-berlin.de}}

\institute[IfI -- HU Berlin]
{Institut für Informatik\\ Humboldt-Universität zu Berlin}



\begin{document}




\frame{
  \titlepage
}


%######################
% Introduction / Setup
%######################

\section{Introduction / Setup}

\frame{
	\begin{block}{Table of Content}

		\begin{itemize}
			\item Introduction
			\item Self-Replicating Hardware
			\item Self-Repairing Hardware
			\item Conclusion
		\end{itemize}

	\end{block}
}


\frame{
	Want?
	\pause

	\begin{itemize}
		\item Mass production of future generations of\\
		      integrated circuits (nanoelectronics)

		\item Achieve John v. Neumanns dream of a self-replicating
		      automata endowed with universal properties of
		      yadda yadda...
	\end{itemize}
	\
	\pause

	Need?
	\pause

	\begin{itemize}
		\item Self-replication to efficiently mass produce

		\item Self-repair to achieve high fault tolerance rate\\
		      (e.g. against production faults)
	\end{itemize}
}


\frame{
	\begin{center}
	How to achieve?
	\end{center}
}


\frame{
	\begin{center}
	Study nature!
	\end{center}
}


\frame{
	Species:
	\pause

	\begin{itemize}
		\item Population level:\\
		      multiple independent organisms, each
		      containing the genome of the entire species
		\ \newline
		\pause


		\item Organism level:\\
		      multiple coorperating cells, each containing the
		      complete DNA but only executes a tiny part of it
		\ \newline
		\pause


		\item Cellular level:\\
		      consists of 'molecular' elements which produce the
		      proteins needed for survial



	\end{itemize}
}


\frame{
	Multicellular organisms:
	\pause

	\begin{itemize}
		\item consist of a finite number of cells, each belonging to
		      certain group of 'specialists'
		\ \newline
		\pause

		\item are build, starting with one cell and using cellular
		      division to construct new cells as copies of the
		      original
		\ \newline
		\pause

		\item most cells have a unique function that characterizes
		      them, but some don't and can virtually become anything\\
		      (nature's void pointers)
	\end{itemize}
}


\frame{
	\makebox[\textwidth]{\pgfimage[width=200pt]{4Level.png}}
	\begin{center}
		\tiny http://www-users.york.ac.uk/~gt512/Images/4Level.gif
		      12.01.'13 - 12:20 CET
	\end{center}
}


\frame{
	\textbf{F}ield \textbf{P}rogrammable \textbf{G}ate \textbf{A}rray :\\
	\ \newline
	\makebox[\textwidth]{\pgfimage[width=220pt]{fpga.png}}
	\footnote{
		\tiny http://upload.wikimedia.org/commons/6/6b/Logic\_block2.svg
		\ 26.01.'13 - 12:31 CET
	}
}


%####################
% Self - Replication
%####################

\section{Self-Replication}


\frame{
	\begin{center}
		\makebox[\textwidth]{\pgfimage[width=100pt]{drawing_hands.jpg}}
	\end{center}

	\begin{itemize}
	\item Self Replication of organisms using 'cellular division'
	\ \newline
	\item Properties of cells:
	\begin{itemize}
		\item each cell contains the whole program (genome)
		\ \newline

		\item each cell knows it's position inside the organism
		\ \newline

		\item only executes part of the genome determined by it's
		      position
		\ \newline

		\item coordinates of new cells achieved through cycling
		       (modulo)
	\end{itemize}
	\end{itemize}
}


\frame{
	\begin{center}
		\makebox[\textwidth]{\pgfimage[width=300pt]{cell_coords.png}}

		\ \newline
		\ \newline

		\tiny http://lslwww.epfl.ch/pages/embryonics/thesis/Thesis-11.gif
		      12.01.'13 - 12:27 CET
	\end{center}
}

\frame{
	How to realize 'cellular division', which in our case means,\\
	copying the program of one cell into others?
}


\frame{
	Multi-Molecular organization of a cell:
	\begin{itemize}
		\item each cell consists of a finite number of molecules
		\ \newline
		\pause

		\item a molecule is a new type of 'FPGA'\\
		      (field programmable gate array)
		\ \newline
		\pause

		\item the logical function of each molecule can be programmed
		\ \newline
		\pause

		\item the sum the functions of the molecules make up the cell
		      and the specific part of the overall genome this cell
		      executes
	\end{itemize}
}



\frame{
	\begin{center}
		\makebox[\textwidth]{\pgfimage[width=200pt]{molcode2.jpg}}
	\end{center}
}




%#################
% Self - Repair
%#################


\section{Self-Repair}

\frame{
	What we demand of self-repair:
	\begin{itemize}
		\item on-line repair mechanism, so we don't have to 'shutdown'
		      the system
		\ \newline

		\item no centralized control mechanism, all should be handled
		      as local as possible
		\ \newline

		\item self-repair on as much levels as possible\\
		      (population, organism, cell and molecule)
	\end{itemize}
}


\frame{
	\begin{center}
		But how?!
	\end{center}
}


\frame{
	\begin{center}
		By \textbf{not} using self-repair!
	\end{center}
}


\frame{
	\begin{center}
		\makebox[\textwidth]{\pgfimage[width=300pt]{school-of-rock.jpg}}
		\ \newline
		\tiny{\url{http://media.tumblr.com/tumblr_lvc6xhsyWN1qzs894.jpg}
		      - 20.01.'13 - 13:50 CET}
	\end{center}
}


\frame{
	\begin{center}
		Self-Repair for organisms:\\
		\pause
		\textbf{for free}
	\end{center}
}


\frame{
	Self-Repair at the cellular level:
	\begin{itemize}
		\item each cell contains complete copy of the 'genome' and
		      therefore acts as a 'stem cell'
		\ \newline
		\pause

		\item Result:\\
		      every cell could literally replace \textit{any}
		      other cell in the organism, just change it's coordinates
		\ \newline
		\pause

		\item Twist:\\
		      a faulty cell will 'kill' it's entire column
		      of cells\\
		      (to reduce replacement complexity)
		\ \newline
		\pause

		\item 'scaring the grid' triggers recomputation of all of the
		      cells coord's, and the state of the now dead cells need
		      to be recovered/copied
	\end{itemize}
}

\frame{
	Self-Repair at molecular level:
	\begin{itemize}
		\item Downside:\\
		      difficult, because we can't actually 'repair' faulty
		      FPGA's
		\ \newline
		\pause

		\item Upside:\\
		      can program molecules, which means we can
		      control (program!) the level of fault
		      tolerance/robustness of our system
		\ \newline
		\pause

		\item Mechanism:\\
		      shift content of dead molecule to it's
		      righthand neighbour and his content to his right
		      hand neighbour, etc. 'till spare molecule is reached
	\end{itemize}
}

%##########
% Summary
%##########

\section{Summary}

\frame{
	\begin{center}
		\makebox[\textwidth]{\pgfimage[width=180pt]{lesson.jpg}}
		\ \newline
		\tiny{\url{http://cdn.memegenerator.net/instances/250x250/32624089.jpg}
		      - 20.01.'13 - 17:05 CET}
	\end{center}
}

\frame{
	Self-Repair:
	\begin{itemize}
		\item no 'repair', instead use spare elements
		\item no centralized control mechanism, all done local
		\ \newline

		\item on-line self repair means:\\
		      go off-line, reconfigure,
		      go on-line again\\
		      (but without intervention from 'outside')
		\ \newline

		\item mechanism opperates on \textbf{multiple} levels\\
		      (molecular and cellular)
		\ \newline

		\item if molecules can't be repaired, trigger cellular
		      self repair:
			\begin{itemize}
				\item  trigger column death of cells via
			               'KILL' signal
				\item  recompute cell coordinates
			\end{itemize}

	\end{itemize}
}


\frame{
	Self-Replication:
	\begin{itemize}
		\item no 'replication', FGPA's need to already exist but
		      are 'empty'
		\ \newline

		\item 'molecules' and 'cells' are connected, so they can
		      propagate data among them
		\ \newline

		\item we construct the data to be propagated and therefor
		      'program' an amount of molecules, which then clones
		      itself, leading to clones of cells and ultimately
		      clones of organism
		\ \newline

		\item process continues untill we run out of 'space'
		\ \newline

		\item creates identical clones, no FPGA mutation yet

	\end{itemize}
}


\frame{
\textbf{Main Source:}
\begin{itemize}
	\item Embryonics Approach:
	\item http://lslwww.epfl.ch/pages/embryonics/thesis/
\end{itemize}
\ \newline

\textbf{Other Projects:}
\begin{itemize}
	\item Field Programmable Gate Array (FPGA) for Bio-inspired
	      visuo-motor control systems applied to Micro-Air Vehicles
	\item http://www.intechopen.com/download/pdf/5965
	\ \newline

	\item On-chip visual perception of motion:\\
	      A bio-inspired connectionist model on FPGA
	\item www.loria.fr/~girau/Publis/NN.pdf
\end{itemize}
}





\end{document}
